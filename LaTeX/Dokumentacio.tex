\documentclass{article}
\usepackage[utf8]{inputenc}


\title{”Programozási alapismeretek”\\ beadandó feladat:\\ ”ProgAlap beadandó” téma 1. feladat}
\date{2016-11-31}
\author{Készítette: Bárdosi Bence\\ Neptun-azonosító: VY9NJN\\ E-mail: bardosi.bence@gmail.com\\\\ Kurzuskód: IP-08PAEG\\ Gyakorlatvezető neve: Pap Gábor Sándorné}
\renewcommand*\contentsname{Tartalom}


\begin{document}
  \pagenumbering{gobble}
  \maketitle
  \newpage

  \pagenumbering{arabic}
  \tableofcontents
  \newpage

  \section{Felhasználói dokumentáció}
    \subsection{Feladat}
      Egy iskolában egyéni és összetett tanulmányi versenyt tartottak. A versenyekben összesen N tanuló vett részt. A versenyek száma M. Ismerjük versenyenként az induló tanulókat és elért pontszámukat. Az összetett versenyben csak azon tanulók eredményét értékelik, akik az összes egyéni versenyen indultak és elérték a versenyenként adott minimális pontszámot.
      \\
      \\ Készíts programot, amely megadja az egyéni versenyek győzteseinek rangsorát!
    \subsection{Futási környezet}
    \subsection{Használat}
      \subsubsection{A program indítása}
      \subsubsection{A program bemenete}
      \subsubsection{A program kimenete}
      \subsubsection{Minta bemenet és kimenet}
      \subsubsection{Hibalehetőségek}
  \newpage

  \section{Fejlesztői dokumentáció}
    \subsection{Feladat}
      Egy iskolában egyéni és összetett tanulmányi versenyt tartottak. A versenyekben összesen N tanuló vett részt. A versenyek száma M. Ismerjük versenyenként az induló tanulókat és elért pontszámukat. Az összetett versenyben csak azon tanulók eredményét értékelik, akik az összes egyéni versenyen indultak és elérték a versenyenként adott minimális pontszámot.
      \\
      \\ Készíts programot, amely megadja az egyéni versenyek győzteseinek rangsorát!
    \subsection{Specifikáció}
    \subsection{Fejlesztői környezet}
    \subsection{Forráskód}
    \subsection{Megoldás}
      \subsubsection{Programparaméterek}
      \subsubsection{Programfelépítés}
      \subsubsection{Függvénystruktúra}
      \subsubsection{Algoritmus}
      \subsubsection{A kód}
    \subsection{Tesztelés}
      \subsubsection{Érvényes tesztesetek}
      \subsubsection{Érvénytelen tesztesetek}
    \subsection{Fejlesztési lehetőségek}


\end{document}
