\documentclass{article}
\usepackage[utf8]{inputenc}
\usepackage{booktabs}
\usepackage{amsmath}
\usepackage{amsfonts}
\usepackage{amssymb}


\title{”Programozási alapismeretek”\\ beadandó feladat:\\ ”ProgAlap beadandó” téma 1. feladat}
\date{2016-11-31}
\author{Készítette: Bárdosi Bence\\ Neptun-azonosító: VY9NJN\\ E-mail: bardosi.bence@gmail.com\\\\ Kurzuskód: IP-08PAEG\\ Gyakorlatvezető neve: Pap Gábor Sándorné}
\renewcommand*\contentsname{Tartalom}


\begin{document}
  \pagenumbering{gobble}
  \maketitle
  \newpage

  \pagenumbering{arabic}
  \tableofcontents
  \newpage

  \section{Felhasználói dokumentáció}
    \subsection{Feladat}
      Egy iskolában egyéni és összetett tanulmányi versenyt tartottak.
      A versenyekben összesen N tanuló vett részt. A versenyek száma M.
      Ismerjük versenyenként az induló tanulókat és elért pontszámukat.
      Az összetett versenyben csak azon tanulók eredményét értékelik,
      akik az összes egyéni versenyen indultak és elérték a versenyenként adott minimális pontszámot.
      \\
      \\ Készíts programot, amely megadja az egyéni versenyek győzteseinek rangsorát!
    \subsection{Futási környezet}
      IBM PC, exe futtatására alkalmas, 32-bites operációs rendszer (pl. Windows 10). Nem igényel egeret.
    \subsection{Használat}
      \subsubsection{A program indítása}
        A program az VY9NJN\textbackslash bin\textbackslash Release\textbackslash VY9NJN.exe néven található a tömörített állományban. A VY9NJN.exe fájl kiválasztásával indítható.
      \subsubsection{A program bemenete}
        A program az adatokat a billentyűzetről olvassa be a következő sorrendben:
        \\
        \begin{table}[h!]
          \centering
          \caption{Bemenet}
          \label{tab:table1}
          \begin{tabular}{ccc}
            \toprule
            \# & Adat & Magyarázat \\
            \midrule
            1 & N & Tanulók száma $(1 \leqslant N \leqslant 100)$ \\
            2 & M & Versenyek száma $(1 \leqslant M \leqslant 100)$ \\
            3 & Min_1 & Az 1. verseny minimum ponthatára $(0 \leqslant Min_1 \leqslant 50)$ \\
            4 & Min_2 & A 2. verseny minimum ponthatára $(0 \leqslant Min_2 \leqslant 50)$ \\
            . \\
            . \\
            . \\
            M+2 & Min_M & Az M. verseny minimum ponthatára $(0 \leqslant Min_M \leqslant 50)$ \\
            \bottomrule
          \end{tabular}
        \end{table}
      \subsubsection{A program kimenete}
        A program kiírja az egyéni versenyek győzteseinek rangsorát.
        A kimenet első sorába az egyéni győzelmet elért tanulók számát,
        amelyet győztesek sorszáma követi, győzelmek száma szerint csökkenő,
        azon belül sorszám szerint növekvő sorrendben.
      \subsubsection{Minta bemenet és kimenet}
      \subsubsection{Hibalehetőségek}
        Az egyes bemeneti adatokat a fenti mintának megfelelően kell megadni.
        Hiba, ha a mérések száma nem egész szám, vagy nem esik a 2..10 000 intervallumba;
        vagy valamely magassági érték nem szám, vagy nem esik a 0..9 000 intervallumba.
        Hiba esetén a program azzal jelzi a hibát, hogy újrakérdezi azt.

  \newpage

  \section{Fejlesztői dokumentáció}
    \subsection{Feladat}
      Egy iskolában egyéni és összetett tanulmányi versenyt tartottak. A versenyekben összesen N tanuló vett részt. A versenyek száma M. Ismerjük versenyenként az induló tanulókat és elért pontszámukat. Az összetett versenyben csak azon tanulók eredményét értékelik, akik az összes egyéni versenyen indultak és elérték a versenyenként adott minimális pontszámot.
      \\
      \\ Készíts programot, amely megadja az egyéni versenyek győzteseinek rangsorát!
    \subsection{Specifikáció}
    \subsection{Fejlesztői környezet}
    \subsection{Forráskód}
    \subsection{Megoldás}
      \subsubsection{Programparaméterek}
      \subsubsection{Programfelépítés}
      \subsubsection{Függvénystruktúra}
      \subsubsection{Algoritmus}
      \subsubsection{A kód}
    \subsection{Tesztelés}
      \subsubsection{Érvényes tesztesetek}
      \subsubsection{Érvénytelen tesztesetek}
    \subsection{Fejlesztési lehetőségek}


\end{document}
